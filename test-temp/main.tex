%
%  =======================================================================
%  ····Y88b···d88P················888b·····d888·d8b·······················
%  ·····Y88b·d88P·················8888b···d8888·Y8P·······················
%  ······Y88o88P··················88888b·d88888···························
%  ·······Y888P··8888b···88888b···888Y88888P888·888·88888b·····d88b·······
%  ········888······"88b·888·"88b·888·Y888P·888·888·888·"88b·d88P"88b·····
%  ········888···d888888·888··888·888··Y8P··888·888·888··888·888··888·····
%  ········888··888··888·888··888·888···"···888·888·888··888·Y88b·888·····
%  ········888··"Y888888·888··888·888·······888·888·888··888··"Y88888·····
%  ·······························································888·····
%  ··························································Y8b·d88P·····
%  ···························································"Y88P"······
%  =======================================================================
% 
%  -----------------------------------------------------------------------
% Author       : 焱铭
% Date         : 2024-02-28 23:14:41 +0800
% LastEditTime : 2024-02-29 19:44:18 +0800
% Github       : https://github.com/YanMing-lxb/
% FilePath     : /PyTeXMK/tests/main.tex
% Description  : 
%  -----------------------------------------------------------------------
%

\documentclass{article} % 指定文档类型为文章
\usepackage[UTF8]{ctex} % 加载ctex宏包以支持中文排版

% 导入必要的宏包
\usepackage{graphicx} % 支持插图
\usepackage{caption} % 支持设置图表标题格式
% \usepackage{tocbibind} % 添加参考文献到目录
\usepackage{nomencl} % 符号索引

% 设置符号索引标题
\renewcommand{\nomname}{符号索引}

\makenomenclature % 创建符号索引

% 示例符号定义
\nomenclature{$c$}{光速}
\nomenclature{$h$}{普朗克常数}
\nomenclature{$G$}{引力常数}

\title{我的第一个LaTeX文档} % 设置文档标题
\author{测试者} % 设置作者姓名
\date{\today} % 设置文档日期

\begin{document}

\maketitle % 生成标题页

% 插入目录、图目录和表目录
\tableofcontents % 目录
\listoffigures % 图目录
\listoftables % 表目录

\section{引言}
LaTeX 是一种基于ΤΕΧ的排版系统,被广泛用于生成高印刷质量的科技和数学文档。它允许用户专注于内容的编写,而不用担心页面布局和格式问题。

\section{LaTeX 基本用法}
LaTeX 使用特殊的标记语言来格式化文本。例如,\textbf{粗体}、\textit{斜体}和\texttt{计算机字体}可以用以下方式实现:

这是一些\textbf{粗体文本},这是一些\textit{斜体文本},以及一些\texttt{计算机字体文本}。

\section{数学公式}
LaTeX 非常适合编写数学公式。例如:

\[ e^{i\pi} + 1 = 0 \]

这是欧拉公式,它被认为是数学中最美丽的公式之一。

% 插入图表示例
\section{图示例}
\begin{figure}[htbp]
    \centering
    \includegraphics[width=0.5\textwidth]{example-image-a}
    \caption{这是一个示例图}
    \label{fig:example}
\end{figure}

\section{表示例}
\begin{table}[htbp]
    \centering
    \begin{tabular}{|c|c|}
        \hline
        姓名 & 分数 \\
        \hline
        小明 & 85 \\
        小红 & 92 \\
        \hline
    \end{tabular}
    \caption{这是一个示例表}
    \label{tab:example}
\end{table}

% 插入参考文献
\section{参考文献}
\nocite{*}
\bibliographystyle{plain}
\bibliography{bibtex-references}

% 生成符号索引
\printnomenclature

\end{document}
